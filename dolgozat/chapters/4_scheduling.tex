\Chapter{Az ütemezési probléma}

\Section{Az ütmezési probléma}

Az ütemezési probléma során azzal a kérdéssel állunk szembe, hogy milyen módon oldható meg, hogy a szabadon rendelkezésre álló programjainkat úgy osszuk be, hogy azzal a lehető leghatékonyabb eredményt érjük el. A lényeg, hogy olyan időbeosztást kapjunk, ahol a nélkülözhető és időt fecsérlő, kevésbé fontos feladatok minél hátrább, míg a relevánsabbak minél hamarabb sorra kerülnek, figyelembe véve az időtartamot is, valamint azt, hogy hogyan fér bele a heti tervezetbe a szokásos napirendünk és a megbeszélt programjaink mellet.

Ahhoz, hogy ezeket valamilyen sorrendben el tudjuk helyezni, és eldöntsük melyik feladattal is kezdjünk és melyiket hagyjuk utoljára, szükségünk van olyan adatokra, amelyekkel jellemezni tudjuk ezeket a tevékenységeket és valamilyen számításra, képletre, amely hozzájárul ezek ütemezéséhez. Ezen adatok melyek jellemzik a tevékenyégeket a már említett prioritás és idő hányados lesz. 

Kell egy olyat matematikai modellt készíteni, amelyet matematikai módszerrel meg tudunk oldani. A prioritás egy fontos tényező, de az idő is meghatározó szerepet játszik. Kétségkívül, ha egy feladat 2 perc alatt letudható, azt érdemes minél hamarabb lerendezni – ld. kétperces technika módszere -, de nem hagyható figyelmen kívül a rövid „to-do-k” miatt egy sok időt felemésztő, de közelgő beadási határidővel rendelkező tevékenység sem.

Ha tehát megegyezik az időtartam kettő vagy több elintéznivalónál, akkor a prioritás fog előtérbe kerülni és amelyik fontosabb azt fogjuk hamarabb elvégezni, míg, ha a prioriás megegyezik, de valamelyiknek rövidebb az ideje, az kerül majd előrébb a sorban. Abban az esetben, ha megegyeznek az időtartamok és a prioritások is, és ugyanazt a hányadost kapjuk a képlet végén, a felviteli sorrend szerint ütemezhetjük be őket. Megjegyezném, hogy egy későbbi, továbbfejlesztett verzióban itt remek lehetőség adódik arra, hogy a modellben ne csak az adott két hányadossal számoljunk, hanem figyelembe vegyük mennyi haszonnal jár vagy épp mennyire nehéz a feladat elvégzése és ezeket is belekalkuláljuk a még hatékonyabb megoldás érdekében.

%ezt az előző fejezetben is leírtam, de nem tudom itt jobb-e megemlíteni, vagy ott

Ha megvan, mi alapján számoljuk ki a hányadosunkat, azt a számítást el kell végezni az összes tevekénységre majd sorba állítani őket. Ezek után már csak sorba be kell illesztenünk őket az üres helyekre, a napirendünk és programjaink mellé. Amennyiben egy nap betelt, tervezhetünk a következőre. Fontos lehet megjegyezni, hogy szükség lehet egyes esetekben a tevékenységek felcserélésére a sorrendben, hiszen ha a sorban következő feladatra már nem jut elég időnk egy nap, de a következő feladatra még épp elég időt tudunk szánni rövidebb időtartama miatt, célszerű lesz az ütemezésbe ilyen módon belejavítani.

A következőkben különböző lehetőségeket vizsgálok meg, amelyek segíthetnek a feladatok beütemezésében és az optimalizált eredmény elérésében.

\Section{A hátizsák feladat}

A hátizsák feladat részletezése:

%órai jegyzet alapján

%...

A hátizsák feladat felhasználása célravezetőnek tűnik, hiszen az érték/tömeg páros teljes mértékben megfeleltethető az általam tervezett alkalmazás prioritás/időtartam attribútomokkal, valamint egy csökkenő sorrendet kapunk eredményül, tehát minél hasznosabb lenne számunka a tevékenység, annál előrébb kerülne a sorban.

%ez lehet csak nekem tűnik kevésnek, de egyelőre nem tudom mit kéne még hozzáfűznöm

%mohó algoritmus? ezt épp csak most láttam, hogy a hátizsák problémát megoldja, akkor részletezzem azt is majd? vagy nem fontos?

\Section{Erőforrás tervezéshez kapcsolódó algoritmusok és szabályok}

A hátizsák feladat ugyan egy az egyben ráilleszthető az adott program matematikai modelljére, mint megoldás, de hasznosnak tartom áttekinteni más lehetőségeket is. A termelésinformatika területén belül, pontosabban erőforrás tervezés során már sokat tanulhattam magáról az ütemezésről, különböző ütemtervek megvalósításáról és a hozzá kapcsolódó algoritmusokról is. A következőkben ezek közül nézek meg néhányat. Érdemes lehet megjegyezni, hogy az erőforrás tervezés során az algoritmusok és az elvégzendő munkák lehetnek egy vagy több gépre tervezve és ezekhez tartozhatnak különböző szabályok az ütemezések célja szerint.

%Kulcsár diáit behivatkozni

\SubSection{Palmer-módszer}

Elsőként a Palmer módszer jutott a témával kapcsolatosan eszembe, amely azért keltette fel a figyelmem, mert itt a munkákhoz prioritási indexet kell rendelni, és ez a prioritási index értéke alapján történik maga az ütemezés.

Az első lépés ennél a módszernél, hogy minden egyes munkához (amely jelen esetünkben nem munkát, hanem tevékenységet jelentene) hozzárendelünk prioritási indexet. Arra, hogy ezt kiszámoljuk egy formula áll rendelkezésünkre, melynek elve, hogy az a munka kerüljön előre, amelyeknek a megmunkálási idői az első gépeken rövidebbek a többihez képest.

%Palmer módszer képlet - diasorról elég kép vagy le kell írni?

Ezek a feladatok többgépes rendszerekre érvényesek, a képletből is látható, hogy itt maga a műveleti időből számítandó a prioritás, ezáltal, mint kiderült, két dolog miatt se tudnánk ezt a módszert saját példánkban alkalmazni – nem tudunk a gépek számára mit behelyettesíteni és a prioritásunk értékét az idő értéke adja meg.


\SubSection{Dannenbring-módszer}

A következő, Dannenbring-módszer ugyan súlyozási sémát használ, viszont szintén többgépes feladatoknál alkalmazható, így visszatértem más tanult módszerek megvizsgálásához, pontosabban az egyetlen erőforrást tartalmazó ütemezési feladatok megoldásaihoz és onnan próbáltam ötleteket szerezni.

\SubSection{Az SPT szabály}

Az SPT (Shortest Processing Time) szabály lényege, hogy a munkákat a műveleti idők alapján nemcsökkendő sorrendbe rendezzük és ennek megfelelően indítjuk el azokat. Ez jelen esetben nálunk azt jelentené, hogy a tevékenységeket az alapján állítanánk sorba, hogy melyikkel tudunk leghamarabb végezni, viszont így az a probléma áll majd fent, hogy a prioritással nem tudunk foglalkozni.


\SubSection{A WSPT szabály}

A WSPT (Weighted Shortest Processing Time) szabály a legkisebb súlyozott műveleti idejű munkát veszi előre. Itt már szóba jön a súlyozás is, a munkák ideje mellett, tehát elsőre már használhatónak tűnik ez a módszer. A WSPT szabály lényege, hogy minden egyes munka – nálunk még mindig tevékenység – esetében wi/pi hányadosokat képzünk, ahol a wi a Ji munka súlyát, a pi a Ji munka műveleti idejét jelenti. Majd elrendezzük a munkákat a kapott wi/pi hányadosok alapján nemnövekvő sorrendbe és ennek megfelelően indítjuk el azokat.  Tehát egy wi/pi fontossági mutatót kapunk és minél nagyobb lesz egy munka mutatója, annál előrébb fog kerülni a sorban.

Ez a szabály láthatóan egy az egyben megfeleltethető a hátizsák feladat példájával, így tehát egy az egyben ráilleszthető a mi problémánkra is.
