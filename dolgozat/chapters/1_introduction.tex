\Chapter{Bevezetés}

Egyetemistaként sokszor találkozhatunk azzal a problémával, hogy alkalomadtán a világ összes ideje sem lenne elég ahhoz, hogy befejezzünk egy-egy beadandót határidőre, megtanuljuk a tananyagot a vizsga napjára vagy éppen eleget aludjunk az előbb felsoroltak teljesítése mellett. Arról nem is beszélve, hogy időnk egy jelentős részét ebben az időszakban gyakran a bulizások, diákmunka végzése vagy az állandó ingázás teszi ki a tanulás mellett. A megfelelő időbeosztás kialakításának kérdése tehát releváns szerepet játszik a hallgatók életében.

Manapság számos módszer áll rendelkezésre, amely hozzájárul feladataink hatékony rendszerezéséhez és a hatékony munkavégzéshez. Ilyen például az Eisenhower mátrix vagy a Pomodoro technika, de számtalan alkalmazás is készült már erre a célra. Ezek az alkalmazások azonban automatikus rendszerezésre nem alkalmasak, csupán egy felhasználói felületet nyújtanak feladatok nyilvántartásához és kezeléséhez.

A szakdolgozatom célja egy olyan alkalmazás készítése, amely egy optimális időbeosztás kialakítását segíti, mely által szabadidőnket a leghatékonyabban használhatjuk ki. Ennek elérése érdekében megvizsgálom a hasonló alkalmazásokat és az elterjedt hatékonyságnövelő módszereknek is utána járok. A megvalósítás során pedig olyan ütemezés felhasználását részesítem előnyben, amely figyelembe veszi és megfelelően kezeli a feladatok fontossága mellett a rájuk szánt időtartamokat is.