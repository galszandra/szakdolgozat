\Chapter{Bevezetés}

Egyetemistaként sokszor találkozhatunk azzal a problémával, hogy alkalomadtán a világ összes ideje sem lenne elég ahhoz, hogy befejezzünk egy-egy beadandót határidőre, megtanuljuk a tananyagot a vizsga napjára vagy éppen eleget aludjunk az előbb felsoroltak teljesítése mellett. Arról nem is beszélve, hogy az adott egyetemista életmód velejárója lehet a gyakori bulizás, a diákmunka vagy esetleg az állandó ingázás. A hatékony időbeosztás kérdése azonban nem csak a hallgatók életében játszhat és játszik is szerepet. Onnantól kezdve, hogy lehetőségünk adódik saját magunk eldönteni, mihez kezdünk a szabadidőnkkel, egészen addig, míg felnőttként, állandó munkahely mellett rendszerezzük a napi teendőinket, nehézségekbe ütközhetünk annak megvalósításában, hogy mindenre jusson időnk és ne kezdjünk instant halogatásba meglátva a napok vagy akár hetek óta felhalmozódott, ránk váró feladatokat.
Manapság már számos módszer áll rendelkezésre, amely valamilyen formában segít a hatékonyságunk növelésében. Ilyen például az Eisenhower mátrix vagy a Pomodoro technika, de különböző alkalmazásokat is használhatunk teendőink nyilvántartására. A leggyakrabban használt, úgynevezett to-do list-eknél, mi magunk jegyezzük fel a jövőben elintézendő feladatainkat, mi rendszerezzük őket, és mi döntjük el melyiknek mikor kezdünk neki. A szakdolgozatom célja egy olyan alkalmazás elkészítése, ahol nem nekünk kell feladataink rendszerezésével foglalkozni, hanem az általunk létrehozott tevékenységek több szempont szerint -, mint például prioritás, időtartam - heti felosztásban automatikusan optimalizálódnak, hogy a lehető legeredményesebb módon tudjuk felhasználni a rendelkezésre álló időnket.

Az alkalmazás készítésekor fontos szem előtt tartani pár tényezőt. Ilyen például a fix időpontú események, amelyeket egy előre megadott időpontra kell rögzíteni a heti tervezőben. Arra is tekintettel kell lennünk, ha egy feladatnak sürgős határideje van vagy magasabb prioritással rendelkezik más tevékenységekkel szemben. A megvalósítás során olyan ütemezést felhasználását részesítem előnyben, amely ezeket a tényezőket figyelembe veszi és megfelelően kezeli.
