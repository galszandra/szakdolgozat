\Chapter{Hatékonyságnövelő módszerek}

A témám kifejtése előtt érdemesnek tartom átnézni azon módszereket, amelyek hozzájárulnak a hatékony időbeosztás megvalósításához, illetve a már meglévő alkalmazásokat, hogy azok milyen funkciókat látnak el. De mit is takar ez az egész időgazdálkodás pontosan és miért van erre szükség?

Az időgazdálkodás a rendelkezésünkre álló idő felhasználására vonatkozó döntések meghozatala, a legfontosabb teendőink elvégzéséhez szükséges idő biztosítása, továbbá a kapcsolatos információk és iratok kezelése, valamint rendelkezésre állásuk biztosítása. (http://docplayer.hu/25267846-Az-idogazdalkodas-alapjai.html) A jó időgazdálkodás lehetővé teszi, hogy ne keményebben, hanem okosabban dolgozzunk.

% TODO: (https://www.mindtools.com/pages/article/newHTE_00.htm)

Két tévhit is működik bennünk az idővel kapcsolatban. Az egyik az, hogy valamikor, egyszer, a jövőben több időnk lesz. A másik pedig az, hogy időt tudunk megtakarítani.  Valójában azonban nem megtakarítjuk, azaz csökkentjük magát az időt, az idő akkor is eltelik, ha nem teszünk semmit, és akkor is, ha sokat teszünk. Amit befolyásolni tudunk: hogyan, milyen tevékenységekkel töltjük el időnket, és azt is, hogyan éljük meg azokat a dolgokat, amelyekre időt fordítunk. 

% TODO: (Hatékony időgazdálkodás - Tananyagszerző: Máthé Judit)

Miért jó az időgazdálkodás?

Számos tanulmány kimutatta, hogy a legjobban megfizetett és a ranglétrán leggyorsabban emelkedő emberek mind-mind rendelkeznek egy tulajdonsággal, amit úgy hívnak, hogy "cselekvésorientáltság". Minden cselekvésüket áthatja ez a szemlélet. A sikeres és hatékony emberek azok, akik mindig rögtön a fontos feladatokkal kezdenek foglalkozni, és addig nem hagyják ezt abba, amíg be nem fejezték az adott feladatot. (Brian Tracy - Személyes hatékonyság c. könyve alapján)
Tehát, megállapíthatjuk, hogy életünkben jelentős előnyei vannak időnk megfelelő menedzselésének, mint például:
\begin{itemize}
\item a nagyobb produktivitás és hatékonyság elérése,
\item jobb szakmai hírnév,
\item kevesebb stressz,
\item nagyobb valószínűség az előrelépésre,
\item nagyobb valószínűség az élet- és karriercélok elérésére,
\item a mindennapi kötelezettségek könnyebb összeegyeztetése.
\end{itemize}

% TODO: (https://www.mindtools.com/pages/article/newHTE_00.htm)

\Section{Hatékonyságnövelő módszerek}

Nem újkeletű téma az, hogy hogyan lehet több időt nyerni magunknak; a régi időkben is sokakat foglalkoztatott már ez a kérdés. Számtalan módszert dolgoztak ki, megannyi technikáról olvashatunk ezzel a témakörrel kapcsolatban. A következőkben ezek közül tekintek át néhány ismertebb megoldást.

\SubSection{Eisenhower módszer}

Az egyik legismertebb technikát, az Eisenhower módszert - más néven prioritás mátrixot -, Dwight D. Eisenhower korábbi amerikai elnök dolgozta ki. Lényege, hogy teendőinket fontosságuk és sürgősségük alapján kategorizáljuk be.

% TODO: Táblázatot beszerkeszteni a forrás alapján! https://www.eisenhower.me/eisenhower-matrix/ ..és behivatkozni majd.

Négy csoportot kapunk így, amelyekhez külön utasítások társulnak.
\begin{itemize}
\item Az első negyed a „Csináld meg” kategória, amely azokat a feladatokat tartalmazza, amelyek fontosak és amiket szükséges még aznap, vagy legkésőbb másnap megcsinálni.
\item A második negyed a „Tervezd be” kategória. Ide tartoznak azok a feladatok, amelyek fontosak, viszont kevésbé sürgősek. Ezeket célszerű betáblázni a közeljövőbe.
\item A harmadik negyed azon feladatoknak van fenntartva, amelyeket delegálhatunk, mivel számunkra kevésbé fontosak, de még mindig elég sürgősek. Érdemes nyomon követni ezeket a megbízásokat, hogy később megbizonyosodjunk előrehaladásukról.
\item A negyedik és egyben az utolsó negyed a „Töröld” kategória. Ide tartoznak, azok a feladatok, amelyekkel egyáltalán nem kéne foglalkozni. Ilyenek lehetnek a rossz szokások, például a felesleges internetezéssel való időtöltés.
\end{itemize}
 
\SubSection{Pomodoro technika}

A Pomodoro technika lényege, hogy miután kiválasztottunk egy feladatot, amin dolgozni szeretnénk, 25 percig csak arra fókuszálunk. Miután letelt ez az időtartam, 5 perc szünetet tartunk, majd kezdődhet a következő intervallum. 4 periódus után pedig egy hosszabb, 20-30 perces pihenőidő ajánlott, mielőtt újra munkába lendülnénk.

A Pomodoro sokak körében ismert és használatos hatékonyságnövelő módszer, amely segít abban, hogy egy adott feladatra koncentráljunk és azt teljesítsük, viszont a szakdolgozatomban a heti szintű előretervezés a cél, így nem tartom célszerűnek erre a technikára több figyelmet fordítani.

\SubSection{GTD}

David Allen Hatékonyságnövelés stresszmentesen c. könyv alapján
A Getting Things Done (röviden GTD) módszerét David Allen tanácsadó ismerteti Hatékonyságnövelés stresszmentesen című könyvében. A munkafolyamat 5 lépésből áll, ezek a következők:
\begin{enumerate}
\item Rögzítés
\item Tisztázás
\item Rendszerezés
\item Reflektálás
\item Cselekvés
\end{enumerate}

A rögzítés lépése arra szolgál, hogy ne kelljen mindent állandóan észben tartanunk. Ennélfogva rögzítenünk kell valamilyen módon – akár írásos, akár digitális formában - minden feladatot, amelyet meg kell csinálnunk.

A tisztázás lépését az alábbi ábra szemlélteti.

% TODO: Az alább említett ábrából csinálni egy saját változatot, de behivatkozni a könyvet is!

David Allen Hatékonyságnövelés stresszmentesen 67. oldal

Ha van egy adott „dolog”, elsősorban meg kell győződnünk arról, hogy van-e vele valami teendőnk vele. Ha nincsen, akkor eldöntjük, hogy szemétbe dobható-e, később lesz-e vele valami teendőnk, vagy pedig potenciálisan hasznos információt tartalmaz, amely később még szükséges lehet számunkra.

Amennyiben az adott dologgal kapcsolatban van teendőnk, akkor három lehetőség közül választhatunk. Ha egy teendő két percnél kevesebb időt vesz igénybe, célszerű abban a pillanatban elvégeznünk, amelyben döntünk róla. Ha tovább tart, mint két perc, fel kell tennünk magunkban a kérdést miszerint mi vagyunk-e a legalkalmasabbak erre a feladatra, és ha nem, akkor érdemes egy megfelelő emberre átruházni a feladatot. Ha két percnél hosszabb időt vesz igénybe az adott tevékenység, és mi vagyunk rá a legalkalmasabbak, akkor vagy későbbre halasztjuk vagy a következő lépésre ugrunk a munkafolyamatunkban, tehát amint lehet megcsináljuk.

Érdemes megfigyelni, hogy alapjaiban véve az Eisenhower módszerhez hasonló megközelítések merülnek fel itt is. Elvégre megjelenik a delegálás, továbbá annak mérlegelése, hogy valamit betervezzünk-e a későbbiekre, vagy kezdjünk el vele az adott pillanatban foglalkozni.
A rendszerezés lépése a dolgaink feldolgozásában és értékelésében segítenek, hiszen fontos, hogy fizikai formában is tároljuk és láthatóvá tegyük a különböző kategóriákba tartozó elemeket.

A reflektálás lényege, hogy átnézzük és aktualizáljuk hetente a listáinkat, így azok folyamatosan rendszerezve legyenek, naprakészek maradjanak, könnyebben tudjunk fókuszban maradni.

A cselekvés pedig arról szól, hogy a felállított rendszert bizalommal használjuk, mivel minden döntésünk intuitív.

\SubSection{Kétperces technika}

A kétperces technika különállóan is megállja a helyét a hatékonyságnövelő módszerek között, de amint az előbb megfigyelhettük, a GTD módszerben is hasznosították ezt a stratégiát, megerősítve ezzel azt, hogy a rövidebb idő alatt elvégezhető feladatok sokat nyomnak a latban a prioritási tényező mellett.

\SubSection{Ivy Lee}

% TODO: https://www.businessinsider.com/ivy-lee-method-productivity-2018-9

Ivy Lee módszere egészen 1918-ig nyúlik vissza, amikor is Charles M. Schwab a cége hatékonyságának növelése érdekében alkalmazta őt. A 100 éves stratégia azon alapszik, hogy minden este leírjuk a 6 legfontosabb feladatot, amint másnap el szeretnénk végezni és prioritásuk szerint megjelöljük őket. Másnap a feladatoknak egyesével kezdünk neki; amikor végeztünk eggyel, akkor lépünk a következőre. A stratégia azért működőképes, mert csökkenti az egyszerre túl sok döntés meghozatalával járó nehézségeket és az időspóroláson kívül segít céljaink priorizálásában is.

Nyilvánvalóan, beszélhetnénk még olyan alternatívákról/opciókról, mint a kanban rendszer, a Zen To Done, a POSEC, Pareto 80/20-as szabálya, mindazonáltal egyértelműen megállapítható, hogy a két alappillér, a prioritás és az idő értéke az, amely szerepet játszik a minél hatékonyabb időbeosztás elérésben.


% TODO: (https://szendreiadam.hu/tervezes/idogazdalkodas-modszerek/ - csak ötletek, kifejtés máshonnan, kell-e hivatkozni majd?) -> Szét kell nézni, hogy neki mi a forrása, de egyébként lehet innen is.

\Section{Hatékonyságnövelő alkalmazások}

A különböző módszerek mellett nem szabad elfelejtkeznünk arról, hogy tennivalóink rendszerezésének megkönnyítésére mára már számos alkalmazás és applikáció létrejött. Ezek többsége azért hasznos, mert nem papír alapú, így könnyen tervezhető, átlátható és többségüket magunknál hordozhatjuk 0-24-ben. Ezek közül a legismertebbeket vesszük most számon, hogy megállapítsuk a mai appok milyen mértékben alkalmazzák az előző pontokban említett módszereket és hogyan járulnak hozzá a rendszerezéshez és hatékonyságnöveléshez.

\SubSection{Evernote}

Vitathatatlan, hogy napjaink egyik legismertebb hatékonyságnövelő alkalmazása az Evernote. Alapjáraton ez az alkalmazás nem kifejezetten csak to-do list-ek készítésére használatos, de megoldható vele feladatlisták létrehozása, amelyekhez emlékeztetőket tudunk beállítani és a checkbox-ok lehetőséget adnak az elvégzett teendők megjelölésére. Hátránya, hogy naptár funkciót nem tartalmaz és a szinkronizálás megoldása (pl. más naptár alkalmazásokkal) se egy egyszerű történet, viszont lehetőségünk van különböző sablonok felhasználására, amik között ma már találunk napi, heti, havi és évi tervező sablonokat, de akár szokáskövetőt is. Érdemes lehet megemlíteni, hogy a sablonok között találunk GTD kategóriát is.

\SubSection{Trello}

A Trello egy olyan alkalmazás, amely kimondottan a személyes hatékonyság és az időgazdálkodás javítására szolgál. Különböző úgynevezett Trello-táblákat hozhatunk létre, kiválasztva a számunkra legmegfelelőbb módszert céljaink elérésére.

A személyes termelékenységi rendszer egy heti feladatrendszerező megoldás, amely segít a feladatlisták kézbentartásában és teendőink áttekintésében.

Használhatjuk az Eisenhower mátrix tábláját is, amely az előző pontokban bemutatott módszert alkalmazza, mely segítségével minden feladat és teendő priorizálható fontosság és sürgősség szempontjából. Ez a tábla segít a hosszú távú célokon tartani a fókuszt, kiszűrve a figyelmet elterelő feladatokat, amelyek miatt nincs idő a lényegi munkára.

A heti felülvizsgálati folyamat során feladatainkat a Teendők, a Folyamatban és a Kész listák használatával tervezzük meg. A heti áttekintő ellenőrzőlisták alkalmazásával könnyedén követhetők a legfontosabb célok és az elért eredmények áttekintése. Ez az alkalmazást David Allen, a GTD módszer megalkotója is ajánlja.

% TODO: hivatkozás: https://trello.com/teams/personal-productivity

\SubSection{Google Calendar}

% TODO: A Google Keep-et is érdemes említeni!

A Google Naptár egy általam is használt és kedvelt alkalmazás. Lehetőségünk adódik akár napi, heti vagy havi nézetben áttekinteni eseményeinket és teendőinket. Adott időpontra vagy egész napra vonatkozóan adhatjuk meg az eseményeket, színjelöléseket használhatunk, és úgynevezett saját naptárakat is alkothatunk, ha egy külön projekthez szeretnénk eseményeket rendelni. Ezeket a naptárakat láthatóvá és nem láthatóvá tehetjük, így külön csoportokban jól áttekinthetővé válik minden esemény. Emlékeztetőket és feladatokat lehet létrehozni, az emlékeztetőknél beállítható az ismétlődés naponta, havonta vagy évente, a feladatokhoz pedig megjegyzéseket csatolhatunk.

\SubSection{Todoist}

A Todoist rendszere önmagában is működőképes, de összeköthető az Evernote-tal, a Trello-val, a Google Calendar-ral és egyéb programokkal is. Funkciói közé tartozik a projektek létrehozása, amikhez feladatokat társíthatunk, projektekbe ágyazott projekteket hozhatunk létre, ha több szálon futó munkáról van szó, illetve határidőket és emlékeztetőket állíthatunk be. Természetesen adhatunk meg prioritást és a színkódolás is megjelenik. Az alkalmazást gyakran használják a GTD elveire építve.

% TODO: hivatkozás: https://www.thecoffeebreak.hu/todoist-bemutato/

A Microsoft To-do segít abban, hogy a különböző tevékenységeket egy helyen vezessük. A szerteágazó feladatok rendszerezésére külön listákat alkothatunk, és ezeket meg tudjuk osztani akár családtagokkal, akár csapattagokkal. Ütemterveket tudunk készíteni, feladatokat kiosztani egy-egy csapattagnak, így nyomonkövetve a tevékenységek folyamatának alakulását.

% TODO: Megnézni, hogy hogy hivatkozható: https://www.youtube.com/watch?time_continue=121&v=-1yH5I7jdvs&feature=emb_title

Az alkalmazások táblázatos összehasonlítását az alábbi ábra mutatja.

% TODO: Az összehasonlító táblázatot beszerkeszteni!

Megfigyelhető, hogy az alkalmazások egy része vagy egy adott hatékonyságnövelő módszerre épül, vagy lehetőségünk van bizonyos szinten megvalósítani, felhasználni benne valamelyiket. A fenti áttekintésből látható az is, hogy funkciók tárháza áll rendelkezésre a rendszerzés megkönnyítése érdekében, de egyik sem nyújt olyan szolgáltatást, amely segítené felhasználóit abban, hogy tippeket adjon a hatékonyabb rendszerezésre vagy automatikusan a legjobb, legoptimálisabb időbeosztást ajánlja fel matematikai képletek vagy számítások alapján. Mivel szakdolgozatomban ezt szeretném megvalósítani, szükséges lesz valamilyen ütemezésre, amelyet a későbbiekben részletezek, de mindenekelőtt a szükséges funkciókat veszem sorra.
