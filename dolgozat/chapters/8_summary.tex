\Chapter{Összefoglalás}

A szakdolgozatomban egy ütemező alkalmazást valósítottam meg, amely egy optimális időbeosztás elérését segíti. A kidolgozás során áttekintettem és összehasonlítottam olyan módszereket és alkalmazásokat, amelyeket a hatékonyságnövelés céljából hoztak létre. Megvizsgáltam különböző algoritmusokat, módszereket és szabályokat, amik az ütemezéssel és optimalizálással kapcsolatos problémákkal foglalkoznak és/vagy azokat oldják meg. A tervezés és megvalósítás során még inkább elmélyedtem a C\# sajátosságaiban, valamint megismerkedtem az \texttt{ItemsControl} és \texttt{Canvas} használatával. Végeredményben sikerült egy olyan alkalmazást létrehoznom, amely prioritás és időtartam alapon vizsgál meg felhasználó által megadott feladatokat, hogy ezáltal egy minél hatékonyabb ütemtervet állítson elő, amelyet grafikus formában, egy diagramon jelenít meg.

A dolgozatban többféle továbbfejlesztési lehetőséget felvetek, közülük mindenképpen a nagyobb volumenű prioritásértékek megadását tartom fontosnak a precízebb eredmény elérése érekében, valamint egy heti szintű megvalósítást, amelyben lehetőség nyílna fix időpontú események megadására, így a program naptárként és ütemezőként is funkcionálhatna egyszerre.

\newpage
\begin{LARGE}
\textbf{Summary}
\end{LARGE}
\vskip 1cm

In my thesis I implemented a scheduling application which aim was to help create an optimized schedule for its users.

First, I reviewed and compared different methods and applications that were developed to increase efficiency. I studied various algorithms, methods and rules that address and/or solve scheduling and optimization problems. During the design and implementation, I got to learn even more about C\# and became familiar with using \texttt{ItemsControl} and \texttt{Canvas}. Ultimately, I was able to create an application that examines the tasks defined by users on the basis of priority and duration, in order to create the most efficient schedule possible and then display it on a diagram.

In the dissertation I proposed several possibilities for further development. I find it very important to specify a higher volume for priority values in order to achieve a more precise result and to create a weekly implementation where users can specify fixed time events so that the application can function as a calendar as well as a scheduler.

