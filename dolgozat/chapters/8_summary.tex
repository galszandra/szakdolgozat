\Chapter{Összefoglalás}

A szakdolgozatomban egy ütemező alkalmazást valósítottam meg, amely egy optimális időbeosztás elérését segíti.

A kidolgozás során áttekintettem és összehasonlítottam olyan módszereket és alkalmazásokat, amelyeket a hatékonyságnövelés céljából hoztak létre. Ennek segítségével következtettem azokra a tényezőkre, amelyeknek figyelembevétele elengedhetetlen volt a megfelelő eredmény kialakítása érdekében. Így esett a választásom a feladatokhoz tartozó prioritás és időtartam értékekre, valamint ezek hányadosaira.

A specifikáció során összegyűjtöttem, hogy milyen adatokat kell majd kezelnie az alkalmazásnak. Ezt követően megterveztem az alkalmazás felhasználói felületét, és leírtam annak elemeit, használati módját.

Az ütemezési probléma kidolgozásához különböző algoritmusokat, módszereket és szabályokat vizsgáltam meg, amelyek az ütemezéssel és optimalizálással kapcsolatos problémákkal foglalkoznak, megoldásokat adnak rájuk. A vizsgálataim alapján végül a hátizsák feladatként emlegetett optimalizálási probléma algoritmusát használtam fel a megvalósítás során, mivel a programom ütemezési problémájához ez illeszkedett a legjobban.

A tervezésnél ismertettem a felhasznált technológiákat, amelyek segítségemre voltak az alkalmazás elkészítésében, valamint összeállítottam a feladatok és hozzá kapcsolódó műveletek modelljét.

Az implementáció során még inkább elmélyedtem a C\# sajátosságaiban, valamint megismerkedtem az \texttt{ItemsControl} és \texttt{Canvas} használatával. Ezek segítségemre voltak abban, hogy a feladatokat és az ütemtervet vizuális formában szemlélteni tudjam. Az ütemezés eredményének megjelenítéséhez a programban az ábrát téglalap és szöveges felirat elemekből készítettem el saját implementációként.

Végeredményben sikerült egy olyan alkalmazást létrehoznom, amely prioritás és időtartam alapján vizsgálja meg felhasználó által megadott teendőket, hogy ezáltal egy minél hatékonyabb ütemtervet állítson elő számára, amelyet grafikus formában, egy diagramon jelenít meg.

A tesztelés során több esetet is megvizsgáltam annak érdekében, hogy megbizonyosodjak a program megfelelő működéséről. Az egyik teszteset során például egy már meglévő ütemterv feladatait használtam újra kisebb módosításokkal, hogy megfigyeljem mennyivel változik az eredmény. A végrehajtott módosítások láthatóan befolyásolták az új ütemterv alakulását, a módosítások mértékének megfelelően.

A dolgozatban többféle továbbfejlesztési lehetőséget felvetettem, közülük mindenképpen a nagyobb volumenű prioritásértékek megadását tartom fontosnak a precízebb eredmény elérése érdekében, valamint egy heti szintű megvalósítást, amelyben lehetőség nyílna fix időpontú események megadására, így a program naptárként és ütemezőként is funkcionálhatna egyszerre.

\newpage
\begin{LARGE}
\textbf{Summary}
\end{LARGE}
\vskip 1cm

In my thesis I implemented a scheduling application which aim is to help to create an optimized schedule for its users.

First, I reviewed and compared different methods and applications that were developed to increase efficiency. In doing so, I identified the factors that were essential to take into account in order to achieve a proper outcome. So I chose the priority and duration values defined for the tasks, and their quotients as these factors.

During specification, I have collected the data needed for my application.
In a further step, I have designed and described the elements and mechanics of the user interface.

To define the scheduling problem, I studied various algorithms, methods and rules that address and/or solve scheduling and optimization problems. Eventually, I decided to use the algorithm of the backpack problem during the implementation as this algorithm suited the scheduling and optimization problem of my program the most.

During the design, I described the used technology which helped me create the application. After, I defined the managed data as a class with their related methods. 

In the implementation phase, I got to learn even more about C\# and became familiar with using \texttt{ItemsControl} and \texttt{Canvas}. These classes helped me to represent the tasks and the schedule in a visual form.
The display of the results is a custom implementation, which based on the drawing of rectangles and font rendering.

Ultimately, I was able to create an application that examines the tasks defined by users on the basis of priority and duration, in order to create the most efficient schedule possible and then display it on a diagram.

During testing, I examined some cases to make sure the program was working properly. In one test case, for example, I reused the tasks of an existing schedule with minor modifications to observe how much different the new schedule would be from the original one. The changes clearly affected the outcome of the new schedule, in line with the extent of the modifications.

In the dissertation I proposed several possibilities for further development. I find it very important to specify a higher volume for priority values in order to achieve a more precise result and to create a weekly implementation where users can specify fixed time events so that the application can function as a calendar as well as a scheduler.
